\section{Introducción}

Este informe presenta el proceso de diseño, construcción y medición de un circuito doblemente sintonizado con adaptación de impedancias en una configuración RLC en paralelo. El objetivo es lograr una frecuencia de resonancia de 18 MHz y asegurar una correcta adaptación de las impedancias tanto en la entrada como en la salida.

Se llevaron a cabo los cálculos necesarios para determinar los valores óptimos de los componentes del circuito, incluyendo los capacitores y un inductor de núcleo de aire, con el propósito de cumplir con los requisitos de frecuencia de resonancia y adaptación de impedancias. Además, se diseñó y fabricó el inductor de núcleo de aire considerando la inductancia necesaria, así como las restricciones de espacio y los materiales disponibles.

Se utilizó Python como ayuda de cálculo numérico y LTSpice para las correspondientes simulaciones.

